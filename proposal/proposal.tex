\documentclass{article}

% Recommended, but optional, packages for figures and better typesetting:
\usepackage{microtype}
\usepackage{graphicx}
\usepackage{subcaption}
\usepackage{booktabs} % for professional tables
\usepackage{fancyhdr}
\usepackage{amsfonts,epsfig}
\usepackage{afterpage}
\usepackage{amsmath,amssymb,amsthm, mathtools}
\usepackage{epsf}
\usepackage{float}
\usepackage{multirow}
\usepackage{mathabx}
\usepackage{dsfont}
\usepackage{amsfonts}
% Seems like ICML style file already includes an algorithm package. We are not
% including this here thus.
%\usepackage[ruled, vlined, linesnumbered]{algorithm2e}
% Add the disable keyword to disable all notes
\usepackage[textsize=tiny, textwidth=1in, disable]{todonotes}
% Tikz from pyplot, plotting
\usepackage{pgf}
\usepackage{graphicx}
\usepackage{caption}
% For custom colors to color the questions
\usepackage{xcolor}
% The \xspace command can be used in \newcommand definitions to make sure there
% is a right amount of space after the command. (So that we don't have to
% manually add tilde like \knn~after each call.
\usepackage{xspace}
% TO remove indent from itemize and in general better itemize/enumerate.
\usepackage{enumitem}
% hyperref makes hyperlinks in the resulting PDF.
% If your build breaks (sometimes temporarily if a hyperlink spans a page)
% please comment out the following usepackage line and replace
% \usepackage{icml2021} with \usepackage[nohyperref]{icml2021} above.
\usepackage{hyperref}
% Custom Macros
\usepackage{mymacros}


% Attempt to make hyperref and algorithmic work together better:
\newcommand{\theHalgorithm}{\arabic{algorithm}}

\newcommand\tian[1]{{\color{orange}\textbf{TL: }{#1} }}

% Use the following line for the initial blind version submitted for review:
\usepackage[accepted]{arxiv}

% If accepted, instead use the following line for the camera-ready submission:
%\usepackage[accepted]{icml2021}

% The \icmltitle you define below is probably too long as a header.
% Therefore, a short form for the running title is supplied here:
%\icmltitlerunning{One-shot Federated Clustering}
\icmltitlerunning{Heterogeneity for the Win: One-Shot Federated Clustering}

\begin{document}

\twocolumn[ 
\icmltitle{Heterogeneity for the Win: One-Shot Federated Clustering} 
% It is OKAY to include author information, even for blind
% submissions: the style file will automatically remove it for you
% unless you've provided the [accepted] option to the icml2021
% package.

% List of affiliations: The first argument should be a (short)
% identifier you will use later to specify author affiliations
% Academic affiliations should list Department, University, City, Region, Country
% Industry affiliations should list Company, City, Region, Country

% You can specify symbols, otherwise they are numbered in order.
% Ideally, you should not use this facility. Affiliations will be numbered
% in order of appearance and this is the preferred way.
\icmlsetsymbol{equal}{*}

\begin{icmlauthorlist}
\icmlauthor{Don Kurian Dennis}{to}
\icmlauthor{Tian Li}{to}
\icmlauthor{Virginia Smith}{to}
\end{icmlauthorlist}

\icmlaffiliation{to}{Carnegie Mellon University, Pittsburgh, PA, USA}

\icmlcorrespondingauthor{Don Dennis}{dondennis@cmu.edu}

% You may provide any keywords that you
% find helpful for describing your paper; these are used to populate
% the "keywords" metadata in the PDF but will not be shown in the document
\icmlkeywords{Machine Learning, ICML}

\vskip 0.3in
]

% this must go after the closing bracket ] following \twocolumn[ ...

% This command actually creates the footnote in the first column
% listing the affiliations and the copyright notice.
% The command takes one argument, which is text to display at the start of the footnote.
% The \icmlEqualContribution command is standard text for equal contribution.
% Remove it (just {}) if you do not need this facility.

\printAffiliationsAndNotice{}  % leave blank if no need to mention equal contribution
%\printAffiliationsAndNotice{\icmlEqualContribution} % otherwise use the standard text.

\begin{abstract}
  \input{sections-arxiv/01-abstract.tex}
\end{abstract}
\input{sections-arxiv/02-intro.tex}
\input{sections-arxiv/04-background.tex}
\input{sections-arxiv/03-method.tex}
\input{sections-arxiv/06-exps.tex}

\bibliography{references}
\bibliographystyle{icml2021}

\clearpage\newpage
\onecolumn
\appendix
\input{sections-arxiv/08-appendixA.tex}


%%%%%%%%%%%%%%%%%%%%%%%%%%%%%%%%%%%%%%%%%%%%%%%%%%%%%%%%%%%%%%%%%%%%%%%%%%%%%%%


\end{document}


% This document was modified from the file originally made available by
% Pat Langley and Andrea Danyluk for ICML-2K. This version was created
% by Iain Murray in 2018, and modified by Alexandre Bouchard in
% 2019 and 2021. Previous contributors include Dan Roy, Lise Getoor and Tobias
% Scheffer, which was slightly modified from the 2010 version by
% Thorsten Joachims & Johannes Fuernkranz, slightly modified from the
% 2009 version by Kiri Wagstaff and Sam Roweis's 2008 version, which is
% slightly modified from Prasad Tadepalli's 2007 version which is a
% lightly changed version of the previous year's version by Andrew
% Moore, which was in turn edited from those of Kristian Kersting and
% Codrina Lauth. Alex Smola contributed to the algorithmic style files.
